\documentclass{article}
\usepackage[T1]{fontenc}
\usepackage[utf8]{inputenc}
\usepackage[brazil]{babel}

\usepackage{amsmath}
\usepackage{listings}
\usepackage{tikz}
\usetikzlibrary{positioning}
\usetikzlibrary{shapes}

\lstset{
    basicstyle=\ttfamily
}

\begin{document}

\title{
    INE5430 --- Inteligência Artificial \\
    FuzzyTruck --- Estacionador Fuzzy de Caminhões
}
\author{
    \begin{tabular}{r l}
        Lucas Berri Cristofolini & 12100757 \\
        Tiago Royer & 12100776 \\
        Wagner Fernando Gascho & 12100779
    \end{tabular}
}

\maketitle

\section{Introdução}

O objetivo deste trabalho era construir um sistema fuzzy
que fosse capaz de estacionar um caminhão.
As entradas do sistema são a posição $(x, y)$ do caminhão
e sua inclinação em relação ao eixo $x$,
e a saída é uma variação nessa inclinação.

Não utilizamos nenhuma biblioteca;
fizemos nosso próprio ``raciocinador fuzzy'' em C++.

Para interfacear com a interface gráfica, feita em Java,
utilizamos os arquivos \lstinline"truck_client.h" e \lstinline"truck_client.cc"
do repositório \lstinline"https://github.com/tarcisiofischer/fuzzytruck".

Nosso código está disponível em \lstinline"https://github.com/royertiago/fuzzytruck".
Instruções de compilação e execução estão no arquivo \lstinline"README.md".

\section{Implementação do raciocinador}

\subsection{Conjuntos fuzzy}

Por simplicidade,
todos os conjuntos fuzzy que utilizamos são trapezoidais.
Eles são representados pela classe \lstinline"FuzzySet".
Cada trapézio é parametrizado pelos quatro valores
$le$, $lm$, $rm$, e $re$,
esquematizados na figura \ref{trapezio}.

\begin{figure}[h]
    \centering
    \begin{tikzpicture}
        \draw[gray] (-1, 0) -- (4, 0);
        \draw[thick] (-1, 0) -- (0, 0) -- (1, 1) -- (2, 1) -- (3, 0) -- (4, 0);
        \draw[dashed] (1, 1) -- (1, 0) (2, 1) -- (2, 0);
        \node at (0, -0.3) {$le$};
        \node at (1, -0.3) {$lm$};
        \node at (2, -0.3) {$rm$};
        \node at (3, -0.3) {$re$};
    \end{tikzpicture}
    \caption{Conjuntos trapezoidais.}
    \label{trapezio}
\end{figure}

Valores que estão entre $lm$ e $rm$ estão totalmente dentro do conjunto representado.
Valores que estão após $re$ ou antes de $le$ estão totalmente fora do conjunto.
O valor de pertinência dos demais valores é obtidos através de interpolações lineares.

Observe que podemos criar conjuntos triangulares escolhendo $lm = re$,
por exemplo.

\subsection{Defuzzificação}

Implementamos o algoritmo descrito no slide $39$ mostrado em aula.
Em essência, dividimos o espaço de $-30$ a $30$ graus em $61$ valores equidistantes;
para cada valor, procuramos qual a regra que ``mais se adequa'' àquele valor,
e fazemos uma média ponderada com estas regras.

\section{Conceitos fuzzy implementados}

Escolhemos dez conjuntos fuzzy para representar os conceitos presentes no problema.

Para o valor de $x$, identificamos três conjuntos importantes:
\lstinline"left", \lstinline"center" e \lstinline"right".
Eles representam, respectivamente,
o fato de o caminhão estar à esquerda da sua vaga,
o fato de o caminhão estar à direita de sua vaga,
e o fato de o caminhão estar verticalmente alinhado à sua vaga.

Para $y$, a distância entre o caminhão e o ``meio-fio'',
fomos menos granulares:
apenas distinguimos entre \lstinline"near" (perto)
e \lstinline"far" (longe).

E, para os ângulos entre o caminhão e o eixo $x$,
separamos cinco conjuntos,
que são usados tanto para entradas quanto para saídas.
Um deles --- \lstinline"a_straight" ---
representa o fato de o caminhão estar perfeitamente na vertical;
dois deles --- \lstinline"a_left" e \lstinline"a_right" ---
identificam pequenos desalinhamentos
e outros dois --- \lstinline"a_very_left" e \lstinline"a_very_right" ---
identificam grandes desalinhamentos.

Também adicionamos um conjunto \lstinline"any",
para representar qualquer situação.
Ela é usada para simplificar algumas regras.

\section{Regras implementadas}

Implementamos 23 regras. Todas elas estão no namespace \lstinline"rules",
no arquivo \lstinline"main.cpp".
Por exemplo, a semãntica da regra
\begin{lstlisting}
    FuzzyRule( left, near, a_straight, a_right )
\end{lstlisting}
é que, caso o caminhão esteja à esquerda, próximo do meio fio,
mas alinhado com a horizontal, ele deve virar à direita.

As regras podem ser divididas em três grupos.
O primeiro grupo (constituído das seis primeiras regras)
lidam com o evento de o caminhão estar muito desalinhado com a vertical;
estas regras tratam de orientá-lo para que fique alinhado.

O segundo grupo (as oito regras seguintes)
dão as instruções para quando o caminhão estiver longe do meio fio.
Neste caso, mesmo que o caminhão esteja pouco desalinhado,
iremos fazer com que o caminhão vire bastante,
para acelerar seu alinhamento com a vertical.

Já o terceiro grupo (as oito últimas regras)
lidam com o caminhão quando ele estiver próximo do meio fio;
procedemos como no grupo anterior,
mas seremos mais ``prudentes'' e não faremos viradas bruscas.

Existe ainda uma regra adicional: caso o caminhão esteja no centro
e alinhado com a vertical, basta ir para frente.

\end{document}
